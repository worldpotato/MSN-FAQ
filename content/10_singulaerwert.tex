\chapter{Singulärwertzerlegung}
\label{chp:singulaerwertzerlegung}
 Eine Singulärwertzerlegung ist die Aufteilung in 3 Teilmatrizen. 
 
 \begin{equation}
 	A = UDV^t
 \end{equation}
 
 \begin{tabular}{cl}
 	U & := eine unitäre  m x m-Matrix \\
 	D & := Diagonalmatrix mit Eignevalues \\
 	V & := Adjungierte einer unitären n x n Matrix
 \end{tabular}

\section{Pseudoinverse}

Eine Pseudoinverse bildet die auf sinuläre Matrizen verallgemeinerte Inverse einer Matrix.

\begin{equation}
A^t := VD^{-1}U^t \; mit \; diag(D^{-1}) = \frac{1}{\sigma})
\end{equation}

Dann heißt die so berechnete Matrix Pseudoinverse $A^t$. 

\section{Anwendung}

\subsection{Nichtsingulär}

Eine Matrix ist genau dann nichtsingulär, wenn 
\begin{equation}
\forall \sigma: \sigma = 0
\end{equation}


\subsection{Schlecht konditioniert}
Eine Matrix ist schlecht konditioniert, wenn

\begin{equation}
\frac{\sigma_{n-1}}{\sigma_0} \approx 0
\end{equation}

Dies hat zur Folge, dass der Gleichungsterm nicht stabil gelöst werden kann.
 
