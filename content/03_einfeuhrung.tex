\chapter{Einführung}
\label{chp:einfuehrung}	

\section{Lokalisieren}
\label{chp:einfuehrung:sec:lokalisieren}
Die Fähigkeit sich gegenüber eines Bezugssystem zu Positionieren.
	
\section{3D Konstruieren}
\label{chp:einfuehrung:sec:3DKonstruieren}
Daten werden immer vom aktuellen Standort aus aufgenommen. Diese können dann nach einer Lokalisierung ins Globale System überführt werden.\\
Wenn Daten von mehreren Positionen aus aufgenommen werden, so muss die Position der Sensoren zueinander bekannt sein.

\section{Mobile Mapping System}
\label{chp:einfuehrung:sec:MobileMappingSystem}

Eigenschaften eines Mobile Mapping Systems:
\begin{enumerate}
	\item Mobile Plattform (Roboter, Flugzeug, Auto, etc.)
	\item Multisensoraufbau zur Vermussung der Umgebung in zwei- oder dreidimensionaler Form
	\item Berechnung des Umgebungsmodels online aber auf offline möglich.
\end{enumerate}
