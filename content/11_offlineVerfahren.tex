\chapter{Offline Verfahren}
\label{chp:offlineVerfahren}

Offline Verfahren kommen dann zum Einsatz wenn eine Echtzeitberechnung nicht notwendig ist. Sie werden nach einer Messreihe verwendet um die Ergebnisse im Nachhinein zu verbessern. \\
\\
Beim Kalmanfilter wird vor allem ein Offline Verfahren verwendet.

\section{Forward/Backward}

Der Forward/Backward Kalmanfilter rechnet die Trajektorie einmal von Vorne und im Anschluss von Hinten. Dadurch werden Fehler ausgeglichen. 

\subsection{Algorithmus}

\begin{enumerate}
	\item Berechnen des Forward Kalmanfilters (Messungen und a Posteriori)
	\item Berechnen der Messwerte und a Posteriori für Rückwärts Rechnung.
	\item Den finalen Messwert berechnen.

\begin{equation}
	Q = P_{forward}^{-1} + P_{backward}^{-1} \\
\end{equation}

\begin{equation}
	x = Q \circ ( P_{forward}^{-1} \circ X_{forward} + P_{backward}^{-1} \circ X_{backward} )
\end{equation}
\end{enumerate}

