\chapter{Range Bearing}
\label{chp:rangeBearing}

Range Bearing Sensoren werden in der Robotik häufig verwendet. Sie geben eine Entfernung und eine Richtung, bezogen auf das Heading, zu einem Messpunkt.

\section{Bewegungsmodell}
\label{chp:rangeBearing:sec:bewegungsmodell}

Auf die aktuelle Pose des Roboters wird die Veränderung addiert. Es wird ein rechtwinkliges Dreieck aufgespannt über das die
Veränderung beschrieben werden kann. Zur Linearisierung werden die Terme partiell abgeleitet zunächst nach $x$, $y$, Winkel (Jacobi Fx) und
dann nach dem Rauschen in Distanz $d$ und Winkel $\alpha$ (Jacobi Fv).

\section{Messmodell}
\label{chp:rangeBearing:sec:messmodell}
Zwischen einer Landmarke und der akutellen Position wird ein rechtwinkliges Dreieck aufgespannt auf dem die Hypothenuse der Distanz
zwischen Position und Landmarke entspricht. Die Orientierung entspricht dem Tangens der beiden Katheten in diesem Dreieck. Zur
Linearisierung werden die Terme partiell abgeleitet nach $x$, $y$, Winkel für das Messmodell $H_x$. Das Messrauchen (hier: $W$) wird außerhalb
des Modells addiert und ergibt nach Linearisierung eine Einheitsmatrix $H_w$