\chapter{Kalmanfilter Gleichungen}
\label{chp:kalmanfilterGleichungen}

\section{Begriffserklärungen}
\begin{description}
	\item[Predicition Phase] Schätzen des Zustandes als Normalverteilung mit dem Bewegungsmodell $F$ (z.B. aus Odometrie) und dem Bewegungsrauschen $Q$
	\item[Measurement update] Durchführung einer Messung $z$ als Normalverteilte Zufallsvariable mit Messrauschen $R$
	\item[Estimation] Berechnung des Kalman Gain $K$ aus geschätztem Zustand und neuer Messung mit Zustandsraummodellierung $H$. Daraus
	berechnet sich neuer Systemzustand $x$ und $P$
	\item[a priori state estimate] Der in Schritt 1 geschätzte Systemzustand \textit{NV}($\bar{x}_k, \bar{P}_k$). Dieser wird \underline{ohne} einer neuen Messung geschätzt.
	\item[a posteriori state estimate] Der in Schritt 3 geschätzte Systemzustand \textit{NV}($\hat{x}_k, \hat{P}_k$). Dieser wird \underline{mit} einer neuen Messung geschätzt.
\end{description}



\section{Gleichungen}

% In dem enumerate sind jeweils die Formeln mit der Legende als tabular
\begin{enumerate}
	\item \textbf{Prediction phase}
	\begin{equation}
	 \bar{x}_k = F \, \hat{x}_{k-1}
	\end{equation}
	
	\begin{tabular}{cl}
	$F$ &:= Bewegungsmodell \\
	$\hat{x}_{k-1}$ &:= Schätzung aus letztem Durchgang
	\end{tabular}
	
	\begin{equation}
	\bar{P}_k = F \, \hat{P}_{k-1} \, F^t + Q_{k-1}
	\end{equation}
	
	\begin{tabular}{cl}
	$\bar{P}_k$ & := a priori state estimate \\
	$\hat{P}_{k-1}$ & := a posteriori state estimate des vorherigen Durchgangs \\
	$Q_{k-1}$ & := Rauschen des vorherigen Durchgangs \\
	\end{tabular}
	
	
	
	
	\item \textbf{Measurement update}
	
	\begin{equation}
		\tilde{z}_k = Messung + \tilde{R}_k
	\end{equation}
	
	\begin{tabular}{cl}
		$\tilde{z}_k$ & := neue Messung \\
		$\tilde{R}_k$ & := Rauschen der Messung
	\end{tabular}
	
	
	

	\item \textbf{Estimation}
	
	\begin{equation}
	K_k = \frac{\bar{P}_k \, H^t_k} {(H_k \, \bar{P}_k \, H^t_k + \tilde{R}_k)}
	\end{equation}
	\begin{tabular}{cl}
		$K_k$ & := neuer Kalman Gain \\
		$H_k$ & := Messmodell
	\end{tabular}


	\begin{equation}
		\hat{x}_k = \bar{x}_k + K_k \, (\tilde{z}_k - H_k \, \bar{x}_k)
	\end{equation}
	In dieser Gleichung werden die zwei Normalverteilungen in eine neue überführt (vgl. \ref{fig:kernideeKalman}). 
	
	\begin{tabular}{cl}
		$H_k \, \bar{x}_k$ & := $\bar{z}_k$ und entspricht damit der Schätzung über das Messmodell\\
	\end{tabular}

	\begin{equation}
		\hat{P}_k = (I - K_k \, H_k)\bar{P}_k
	\end{equation}
	
	
	
\end{enumerate}