\chapter{Cartesian Motion}
\label{chp:cartesianMotion}
Cartesian Motion ist eine Bewegung auf $n$ rechtwinklig zueinander stehenden Achsen.

\section{Glatter Übergang zwischen zwei Koordinatensystemen}
Um den Übergang zwischen Koordinatensystemen zu berechnen muss für die Rotation die RPY Darstellung [\ref{chp:MathematischeGrundlagen:sec:RPY}] verwendet werden. \\
Anschließend wird Komponentenweise interpoliert.


\begin{enumerate}
\item Interpolieren von RPY
\begin{equation}
		\begin{array}{l}
		r^{(i)}(s) := (1 - s) \cdot r^{(0)} + sr^{(N)}
 \\
		p^{(i)}(s) := (1 - s) \cdot p^{(0)} + sp^{(N)}
\\
		y^{(i)}(s) := (1 - s) \cdot y^{(0)} + sy^{(N)}
\\
		\\
		mit \; s := \frac{k}{N+1} \;|\; k \; \in \{1,...,N-1\}
		
		
		\end{array}
\end{equation}

\item Interpolieren des Ursprungs

\begin{equation}
\begin{array}{l}
T_x^{(i)}(s) := (1 - s) \cdot T_x^{(0)} + sT_x^{(N)}
\\
T_y^{(i)}(s) := (1 - s) \cdot T_y^{(0)} + sT_y^{(N)}
\\
T_z^{(i)}(s) := (1 - s) \cdot T_z^{(0)} + sT_z^{(N)} \\
\end{array}
\end{equation} 	
\end{enumerate}


\section{Warum nicht mit 3x3 Matrix}
Da die Eigenschaften der Drehungen verloren gehen würden. Daher:

\begin{enumerate}
	\item Überführung in RPY Darstellung
	\item Überführung in nächstes Koordinatensystem.
	\item Zurück in Rotationsmatrix
\end{enumerate}

\section{Nachteil von RPY Darstellung}
Dabei können Singularitäten auftreten. Bei Pitch = $\pi/2 $ bzw. 90\textdegree) \\
