
\section{Kalmanfilter}
\label{sec:faq:kalmanfilter}


\subsection{Forward/Backward Algorithmus}
Die Nutzung der Varianzen, gerechnet vor vorne und hinten.
\subsection{Singulärwertzerlegung}
 Aufteilung in 3 Teilmatrizen. A = UDVt. U: eine unitäre  m x m-Matrix ist, D: Diagonalmatrix V:  die Adjungierte einer unitären n x n  n x n-Matrix V
\subsection{UWB Messungen, wie kann eine Trojektorie eschätzt werden}
Polynome vom Grad x (p(t) = a0 + a1t + a2t$^2$...) a0 + a1 * 0 + a2 * 0$^2$ + ... = [x0 y0]. Das ganze wird in Matrix-Vektor form gebracht.  Werte in der Struckturmatrix werden werden durch tn geteilt, um sie zu normalisieren. Anschließend in matlab mit ax = linsolve(A, ax)... 
\subsection{Was ist ein homogenes linieares Gleichungssystem}
Homogen: Auf den Nullraum abgebildet. (Rechter und untererer Rand sind 0). Lineares Gleichungssystem: gleich viel gleichungen wie unbekannte.
\subsection{Was ist ein überbestimmtes und inhomogenes gleichungssystem}
Ich bilde nicht auf dne Nullraum ab und habe mehr gleichungen als Unbekannte.

\subsection{Kernidee Integrity Monitoring}

Abweichungen von über 3 Sigma sind nicht zu verwenden.

\subsection{Phasen des Kalmanfilters}
\begin{description}
\item[Prediction Phase] Der Teil beim Kalmanfilter der als erstes ausgeführt wird, es wird das Bewegungsmodel verwendet und geschätzt wo wir sind.
\item[measurement update] die neue Messung kommt rein.
\item[Estimation] eigentlicher Filterschritt.
\end{description}

\subsection{wie lauten die Modellannahmen des Kalman Filters}
Normalverteilte Rauschen, Normalverteilt, Lineares Bewegungsmodel,
\subsection{Warum linear Verteilt}
Damit eine Normalverteilung normal bleibt. 

\subsection{warum ist die Modellannahme "linear" beim EKF wichtig?}

Dieser Arbeitet mit nicht linearen Modellen, welche linearisiert werden müssen.

\subsection{Warum Normalverteilungen}
Weil diese deutlich einfacher zu verarbeiten sind.  

\subsection{Was bewirkt der Kalmanfilter}
Verrechnet eine Schätzung aus einem Bewegungsmodell und der aktuellen Messung miteinander und schätzt dadurch den realen Wert.

\subsection{Warum ergibt eine neue Zustandsschätzung wieder eine Normalverteilung}

Weil wir nur Normalverteilungen haben und eine Normalverteilung * Normalverteilung = Normalverteilung
\subsection{Wenn Mittelwert bekannt, wie berechnung der Varian}
In Strucktur der Normalverteilung bringen und dann Varianz ablesen.
\subsection{Cartesian Motion}





\section{Kapitel 5}
\label{sec:faq:kap5}

\newpage

\chapter{Partikelfilter}
\label{kap6Partikelfilter}


	
	\section{WDF}
	Wahrscheinlichkeitsdichtefunktion \cite{wdf}
	
	\section{Mutation}
	Das verändern eines Messwertes von einer Generation zur nächsten
	
	\section{Modellannahmen des Partikelfilters}
	
	\begin{itemize}
		\item Sensoren nicht normalverteilt
		\item WDF kann multimodal / mit mehreren Peaks sein, damit kann ich mich Global positionieren
	\end{itemize}
	

		
	
	\section{Kalmanfilter vs. Partikelfilter}	
		\begin{itemize}
		\item Normalverteilt vs. nicht normalverteilt
		\item Linear vs. nicht linear
		\item Kalmanfilter ist mathematisch, Partikelfilter ist algorithmisch
	\end{itemize}

	\section{Generation}
		Eine Menge von N posen (Partikel) zum Zeitpunkt $X_k$ \\
		Jede Generation approzimiert die Wahrscheinlichkeitsdichtefunktion (WDF) anhand des momentan verfügbaren Wissens

	\section{Partikel}
		Eine Pose der Generation $X_k$
		
	\section{Schritte des Partikelfilter}

		Es gibt einen Init und fünf reguläre Schritte dazu noch einen um die Position zu berechnen
		
		\subsection{Init}
		Initialisierung durch Gleichverteilung von Partikeln
		
		\subsection{Bewegung}
		Bewegen des Roboters \textit{(Zeitpunkt k - 1)}
		
		\subsection{Messung}
		Messung durchführen \textit{(Zeitpunkt k)}
		
		\subsection{Propagation}
		Propagation aller Partikel durch Bewegungsmodell (Zeitpunkt k) (Mutation kann auch hier durchgeführt werden)
		Hier werden die Partikel mit dem Bewegungsmodel bewegt. \\
		Wenn $Q$ (Rauschen) mit 0 angenommen wird, \\
		$Q$ wird verwendet um das Rauschen der Sensoren zu imitieren. Außerdem wird es verwendet um die Partikel zu verteilen.
		
		\subsection{Gewichtung}
		Bewertung der Messungen für jedes Partikel. \\
		\\
		\textbf{Gewichtung Beispiel:} \\
		Differenz zwischen Partikelmessung und realer Messung. Reale Messung als Zentrum einer Normalverteilung und Gewichtung entspricht der Wahrscheinlichkeit in dieser Normalverteilung.
		
		\subsection{Neuverteilung der Partikel}
		Neuverteilung der Partikel mit Mutation
		
		\subsection{Berechnung der Position}	
		Berechnung der Position mittels gewichtetem Mittel oder ransac \cite{ransac} (\ref{uebersicht:sec:RANSAC})

	
	\section{Anzahl Schleifen in Init}
	Wieviel Schleifen in init

