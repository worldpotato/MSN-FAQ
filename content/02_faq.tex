\chapter{FAQ}
\label{cha:faq}

\section{Kapitel 1}
\label{sec:faq:kap1}
\begin{description}
	\item[Homogene Matrix]
 	\item[Pose] Beschreibt die Lage eines Körpers.
	\item[Singulärwertzerlegung] Zerlegung einer Matrix in 3 Spezielle Matrizen welche miteinander Multipliziert die grundlegende Matrix ergeben. Auf der Hauptdiagonalen der mittleren Matrix stehen die Singularitäten der grundlegenden Matrix. 
	\item[Bündelblockausgleichung] Eine Ausgleichung um die Bilder von mehreren Kameras oder einer Kamera aus mehreren Winkeln zusammen zu führen und somit ein Bild zu bekommen. Hat als Ergebnis eine nicht lineare Funktion.
	\item[Levenberg‐Marquardt] Ist ein numerischer Optimierungsalgorithmus zur Lösung nichtlinearer Ausgleichs-Probleme mit Hilfe der Methode der kleinsten Quadrate. 
	\item[Range‐Bearing Sensor (mit Beispielen)] Sensor welcher die Richtung und die Entfernung zu einem Messpunkt angibt.
	\item[Bearing only Sensor (mit Beispielen)] Sensor welcher nur die Richtung angibt.
	\item[Trajektorie] Der Weg eines Objektes.
	\item[Koppelnavigation / dead reckoning] Das aneinanderfügen vergangener Standortmessungen, welche jeweils relativ zum letzten Messzeitpunkt sind.
	\item[Sigma Point Kalman Filter] Kalman Filter für nicht lineare Gleichungssysteme. Legt eine Normalverteilte Punktwolke um den aktuellen Punkt. Stabiler als der Kalmanfilter.
	\item[Extended Kalman Filter] Iterative Lösung.
	\item[ICP oder Scanmatching] Iterative Closest Point um eine Punktwolke mit 
	\item[Partikelfilter] Kann aus dem Vergleich vieler Messungen zu einer bekannten "Karte" den Ort absolut bestimmen.
	\item[RANSAC]Random sample consensus - Eine iterative Methode um outliner zu erkennen oder eine Gerade durch eine Punkwolke zu legen, welche viele outliner hat.
	\item[Loop closure] Wenn Messungen einen geschlossenen loop bilden. Dies kann genutzt werden um Filter und Ausgleichungen zu testen oder zu verbessern.
	\item[SLAM] simultaneous localization and mapping - Zeitgleiches Positionieren und mappen der Messdaten. Wird in unbekannter Umgebung verwendet.
	\item[lokalisieren] Wie kann ich meine Position "Global" berechnen.
	\item[3D Konstruieren] Daten werden immer vom aktuellen Standort aus aufgenommen. Diese können dann nach einer Lokalisierung ins Globale System überführt werden.
	\item[Mobile Mapping] Eine Mobile Plattform mit einer Multisensorplattform nimmt die Daten auf. Die Map kann aber im Postprocessing berechnet werden.
\end{description}

\section{Kapitel 2}
\label{sec:faq:kap2}

\begin{description}
	\item[Skalarprodukt] (Kap.2/2) 
						\begin{equation}
							\vec{a} \circ \vec{b} = |\vec{a}| \cdot |\vec{b}| \cdot \cos\varphi 
						\end{equation}
						\begin{equation}
							\vec{a} \circ \vec{b} =
							\left(
								\begin{array}{c}
									a_{x} \\
									a_{y}
								\end{array}
							\right) 
							\cdot 
							\left(
								\begin{array}{c}
									b_{x} \\
									b_{y}
								\end{array}
							\right)
							=
							a_{x}b_{x}
							+
							a_{y}b_{y} 
						\end{equation}
	\item[Winkel zwischen zwei Vektoren?] (Kap.2/2)
		\begin{equation}
			\cos\varphi = \frac{\vec{a} \circ \vec{b}}{|\vec{a}| \cdot |\vec{b}|}
		\end{equation}	
	\item[zwei Vektoren orthogonal?] (Kap.3)
		\begin{equation}
			\vec{a} \circ \vec{b} = 0
		\end{equation}
	\item[Vektorprodukt?] (Kap. 2/5)
		\begin{equation}
			\vec{a} \times \vec{b} = \vec{c}
		\end{equation}
		\begin{equation}
			|\vec{c}| = |\vec{a}| \cdot |\vec{b}| \cdot \sin\varphi 
			\quad \textrm{for} \quad 
			(\ang{0} \leq \varphi \leq \ang{180})
		\end{equation}
	\item[Zusammenhang "rechter Hand" und Vektorprodukt?] (Kap. 2/5)
		Die Rechte Hand gibt die Richtung der Achsen vor. 
	\item[Matrix/Vektor Form des Vektorproduktes?] (Kap. 2/6)
		\begin{equation}
			\vec{a} \times \vec{b} = 
				\left(
					\begin{array}{c}
						a_{x} \\
						a_{y} \\
						a_{z}
					\end{array}
				\right)
				\times
				\left(
					\begin{array}{c}
						b_{x} \\
						b_{y} \\
						b_{z}				
					\end{array}
				\right)
				= 
				\left(
					\begin{array}{c}
						a_{y}b_{z} - b_{z}a_{y} \\
						a_{z}b_{x} - b_{x}a_{z} \\
						a_{x}b_{y} - b_{y}a_{x}  
					\end{array}
				\right)				
		\end{equation}
	\item[Eigenschaften einer Rotation?] (Kap. 2/8)
		Winkel der Drehung um jeweils eine Achse.
	\item[Unterschied zwischen Rotation und Spiegelung?] (nicht im Skript)
Gegeben sei eine 3 x 3 Matrix. Wie kann überprüft werden, ob es sich um eine Rotation handelt?] (Kap. 2/8)
	\item[Darstellung Translation und Rotation als homogene Matrix?] (Kap. 2/9)
	\item[Vorteile des Rechnen mit homogenen Matrizen?] (nicht im Skript)
	\item[Wie kann eine homogene Matrix als Koordinatensystem interpretiert werden?] (Kap. 2/11)
	\item[Wie lässt sich aus einer homogenen Matrix der Koordinatenursprung, sowie die Achsen berechnen?] (nicht im Skript)
	\item[Was versteht man unter einem affinen, orthogonalem und orientierungstreuem Koordinatensystem?] (Kap. 2/11)
	\item[Was versteht man unter der RPY Darstellung einer Drehung?] (Kap. 2/15)
	\item[Wie lassen sich aus der RPY Darstellung einer Drehung die drei Winkel zurückrechnen?] (Kap. 2/16)
	\item[Welche anderen Darstellungen für Drehungen gibt es noch?] (Kap. 2/21)
\end{description}
\section{Kapitel 3}
\label{sec:faq:kap3}
\begin{description}
 	\item[Wie lauten die Modellanahmen des Kalmanfilters]
	\item[Rauschen ist zufällig] Normalverteilt und nicht systematisch.
	\item[Lineare abbbilungen] Damit die Normalverteilung eine Normalverteilung bleibt.
	\item[der EKF arbeitet nicht linear] Das nicht lineares wird linearisiert. Dabei werden Fehler gemacht.
	\item[In einem Satz: was macht der Kalmanfilter?] Die aktuelle Zugangsschätzung und der neue Messwert haben beide Fehler und die Wahrheit liegt in der Mitte. Aus der Mitte wird der neue Wert der schätzung berechnet.
	\item[warum gibt die neue Zustandsschätzung wieder eine Normalverteilung] Es wird mit Normalverteilungen gerechnet.
	\item[Normalverteilung: Zussamenhang zwischen Extremwert und Mittelwert] Sind identisch.
	\item[Inovation?] Der unterschied zwischen estimation und messung
	\item[Aus welchen Parametern wird der update Gain berechnet] Standardabweichungen
	\item[Wertebereich des Gain] 0-1
	\item[Wie lässt sich der Bereich des Mittelwertes der neuen Schätzung eingrenzen?]
	
\end{description}

\section{Kapitel 4}
\label{sec:faq:kap4}
\begin{description}
	\item[Forward/Backward Algorithmus] Die Nutzung der Varianzen, gerechnet vor vorne und hinten.
	\item[Singulärwertzerlegung] Aufteilung in 3 Teilmatrizen. A = UDVt. U: eine unitäre  m x m-Matrix ist, D: Diagonalmatrix V:  die Adjungierte einer unitären n x n  n x n-Matrix V
\end{description}

\section{Kapitel 5}
\label{sec:faq:kap5}

\section{Kapitel 6}
\label{sec:faq:kap6}

\cite{test}