\chapter{Mathematische Grundlagen}
\label{chp:MathematischeGrundlagen}




\section{Skalarprodukt}
\label{chp:MathematischeGrundlagen:sec:skalarprodukt}

	\begin{equation}
		\vec{a} \circ \vec{b} = |\vec{a}| \cdot |\vec{b}| \cdot \cos\varphi 
	\end{equation}
	\begin{equation}
		\vec{a} \circ \vec{b} =
		\left(
		\begin{array}{c}
		a_{x} \\
		a_{y}
		\end{array}
		\right) 
		\cdot 
		\left(
		\begin{array}{c}
		b_{x} \\
		b_{y}
		\end{array}
		\right)
		=
		a_{x}b_{x}
		+
		a_{y}b_{y} 
	\end{equation}

\section{Winkel zwischen zwei Vektoren}
\label{chp:MathematischeGrundlagen:sec:winkelVektoren}
	\begin{equation}
		\cos\varphi = \frac{\vec{a} \circ \vec{b}}{|\vec{a}| \cdot |\vec{b}|}
	\end{equation}	


\section{Orthogonalität}
\label{chp:MathematischeGrundlagen:sec:Orthogonalität}
	Zwei Vektoren $\vec{a}$ und $\vec{b}$ sind Orthogonal zueinander wenn das Vektorprodukt 0 ergibt.
		\begin{equation}
			\vec{a} \circ \vec{b} = 0
		\end{equation}

\section{Vektorprodukt}
\label{chp:MathematischeGrundlagen:sec:Vektorprodukt}
	\begin{equation}
		\vec{a} \times \vec{b} = \vec{c}
	\end{equation}
	\begin{equation}
		|\vec{c}| = |\vec{a}| \cdot |\vec{b}| \cdot \sin\varphi 
		\quad \textrm{for} \quad 
		(\ang{0} \leq \varphi \leq \ang{180})
	\end{equation}

\subsection{Matrix/Vektorform}

\begin{equation}
\vec{a} \times \vec{b} = 
\left(
\begin{array}{c}
a_{x} \\
a_{y} \\
a_{z}
\end{array}
\right)
\times
\left(
\begin{array}{c}
b_{x} \\
b_{y} \\
b_{z}				
\end{array}
\right)
= 
\left(
\begin{array}{c}
a_{y}b_{z} - b_{z}a_{y} \\
a_{z}b_{x} - b_{x}a_{z} \\
a_{x}b_{y} - b_{y}a_{x}  
\end{array}
\right)				
\end{equation}

\section{Rechte Hand Regel}
\label{chp:MathematischeGrundlagen:sec:RechteHandRegel}
	Die Rechte Hand gibt die Richtung der Achsen vor, die Vektoren (Finger) bilden ein Rechtssystem deren Vektorprodukte alle 0 ergeben.
	\begin{itemize}
		\item Daumen: X
		\item Zeigefinger: Y
		\item Mittelfinger: Z
	\end{itemize}


\section{Rotation}
\label{chp:MathematischeGrundlagen:sec:Rotation}

\subsection{Definition}
\begin{enumerate}
	\item Winkel der Drehung um jeweils eine Achse
	\item Rotationsmatrix multipliziert mit der Transponierten ergibt die Einheitsmatrix $I$
	\item Spalten stehen Senkrecht aufeinander
\end{enumerate}

\subsection{Rotation vs. Spiegelung}

	\subsection{Rotation}
 		überführt ein Rechtwinkliges/Rechtshändiges Koordinatensystem in ein anderes Rechtwinkliges/Rechtshändiges Kooridinatensystem. RxR ist Einheitsmatrix.
 	\subsection{Spiegelung}
		Die Hand wechselt. Die Detimernante wird -1

\subsection{Prüfung auf Rotation}

\begin{lstlisting}[
style      = Matlab-editor,
basicstyle = \mlttfamily,
]
if abs(max(max((rotation * rotation') - eye(3)))) > 0.0000000000001
isRotation = false;
else
isRotation = true;

\end{lstlisting}


\section{Homogene Matrix}
\label{chp:MathematischeGrundlagen:sec:HomogeneMatrix}

\subsection{Translation und Rotation}
\begin{equation}
	M = 
	\left(
	\begin{array}{cc}
		R & T\\
		0 & 1
	\end{array}
	\right)
	=
	\left(
	\begin{array}{cccc}
		r_{11} & r_{12} & r_{13} & t_{1} \\
		r_{21} & r_{22} & r_{23} & t_{2} \\
		r_{31} & r_{32} & r_{33} & t_{3} \\
		0 	   & 0    & 0    & 1  
	\end{array}
	\right)	
\end{equation}

\subsection{Vorteile}
	Mehrere Bewegungen können durch Matrixmultiplikation aneinander gekettet werden bzw. leichter invertiert werden.

\subsection{Intepretation als Koordinatensystem}

Eine Homogene Matrix kann auch als Koordinatensystem interpretiert werden. 



\subsection{Berechnung Koordinatenursprung}

\begin{equation}
T
=
\left(
\begin{array}{c}
t_{1} \\
t_{2} \\
t_{3} \\
\end{array}
\right)	
\end{equation}


\subsection{Berechnung der Achsen}
\begin{equation}
x =
\left(
\begin{array}{c}
r_{11}  \\
r_{21} \\
r_{31} \\
\end{array}
\right),	
y =
\left(
\begin{array}{c}
r_{12}  \\
r_{22} \\
r_{32} \\
\end{array}
\right),	
z =
\left(
\begin{array}{c}
r_{13}  \\
r_{23} \\
r_{33} \\
\end{array}
\right)	
\end{equation}


\section{Koordinatensysteme}
\label{chp:MathematischeGrundlagen:sec:Koordinatensysteme}

\subsection{Affines Koordinatensystem}
Lineare Koordinatensystem

\subsection{Orthogonales Koordinatensytem}
Rechtwinklig

\subsection{Orientierungstreues Koordinatensystem}
Es bleibt nach der rechten Hand definiert.

\section{RPY Darstellung}
\label{chp:MathematischeGrundlagen:sec:RPY}
Jede Rotation kann als Roll, Pitch, Yaw dargestellt werden.

\begin{tabular}{cc}
	Roll & X-Achse \\
	Pitch & Y-Achse \\
	Yaw & Z-Achse
\end{tabular}


\subsection{Rotation als RPY}
Dabei können Singularitäten auftreten. (Bei Pitch = $\pi/2 $ bzw. 90\textdegree) \\
	\textbf{Matrix to RPY}
\begin{lstlisting}[
style      = Matlab-editor,
basicstyle = \mlttfamily,
]
y = atan2(rotation(2,1), rotation(1,1));
p = atan2(-rotation(3,1), rotation(1,1) * cos(y) + rotation(2,1) * sin(y));
r = atan2(rotation(3,2)/cos(p), rotation(3,3)/cos(p));
\end{lstlisting} 

\subsection{RPY to Matrix}
Eine Rotation, welche zuerst um die Roll, dann die Pitch und zuletzt die Yaw achse dreht

\begin{equation}
R_{\alpha, \beta, \gamma} % rpy
=
\left(
\begin{array}{ccc}
\cos{\alpha}\cos{\beta} & \cos{\alpha}\sin{\beta}\sin{\gamma} - \sin{\alpha}\cos{\gamma}  & \cos{\alpha}\sin{\beta}\cos{\gamma} + \sin{\alpha}\cos{\gamma} \\
\sin{\alpha}\cos{\beta} & \sin{\alpha}\sin{\beta}\sin{\gamma} + \cos{\alpha}\cos{\gamma} & \sin{\alpha}\sin{\beta}\cos{\gamma} - \cos{\alpha}\sin{\gamma}  \\
-\sin{\beta}	   & \cos{\beta}\sin{\gamma}    &   \cos{\beta}\cos{\gamma}
\end{array}
\right)	
\end{equation}


\subsection{Weitere Darstellungen}
\begin{itemize}
	\item Quaternionen \cite{quaternionen}
	\item Axis/Angle Darstellung \cite{axis-angle} -> numerisch stabil
\end{itemize}