\chapter{Sigma Point Kalmanfilter}
\label{chp:sigmaPointKalmanfilter}

Der Sigma Point Kalmanfilter benötigt keine Linearisierung des Messmodells, da die Normalverteilung für den nächsten Schritt nicht durch das Messmodell berechnet wird.\\
Um den wahrscheinlichsten Punkt werden Normalverteilt Sigma-Points berechnet. Diese Punkte repräsentieren die Normalverteilung und werden nun mit dem Messmodell bewegt. \\
Durch die nun nicht mehr Normalverteilten wird nun ein neuer Zustand berechnet.

\section{Kernschritte}
\begin{enumerate}
	\item Erstellen der Sigma Points mit Gewichtung
	\item Bewegen der Sigma Points 
	\item Erstellen der neuen Normalverteilung aus Sigma Points mit Gewichtung
\end{enumerate}