\chapter{Trajektorie}
\label{chp:trajektorie}

\section{Definition}
Der Weg eines Objektes in Abhängigkeit von der Zeit.\\
Stellt auch die Lösungskurve einer Differenzialgleichung dar.

\begin{equation}
P_{(t)} = 
\left(
\begin{array}{c}
a_0 + a_1t + a_2t^2 \\
b_0 + b_1t + b_1t^2
\end{array} 
\right)
\end{equation} 

\section{Quintisches Polynom}
Ein Quintisches Polynom enthält 6 Komponenten $\{a_0, a_1, a_2 , ..., a_5\}$ und ist, wie der Name vermuten lässt, vom 5. Grad. \\

\subsection{Interpolations Annahmen}
Es werden $Freiheitsgrad + 1$ Annahmen benötigt. \\
Diese können z.B. Position, Geschwindigkeit und Beschleunigung jeweils für den Start- und End-Punkt.\\
\\
Alternativ könnten auch Position, Range und Bearing jeweils für den Start und End-Punkt als Annahme dienen.

\subsection{Pseudocode Implementierung}
\begin{enumerate}
	\item Erstellen der Strukturmatrix mit Normierter Zeit variable  $A$(1x6)
	\item Erstellen des Ergebnisvektors $b$ (2x1)
	\item Lineares Gleichungssystem lösen
	\item Zeitpunkt bestimmen, an dem die Position bestimmt werden soll.
	\item Zeitpunkt in Gleichungssystem einsetzen
\end{enumerate}

\subsection{Überbestimmung}
Wenn es mehr Interpolationsbedingungen als Freiheitsgrade gibt ist es nicht mehr eindeutig zu lösen. Dies führt zu einer Unsicherheit.

\subsection{Vorteile des Quintischen Polynoms}
\begin{itemize}
	\item Beschreiben der Trajektorie speicherarm als Mathematische Funktion.
	\item Einfaches Interpolieren
\end{itemize}

\subsection{Vorteile von Endpunkt $T = 1$}
Das Polynom bleibt dadurch numerisch stabiler.

